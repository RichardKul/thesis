%-englische-Zusammenfassung---------------------------------------

%\selectlanguage{english}

%\begin{abstract}
%\setcounter{page}{2} % Nach Bedarf anpassen!
%Here is the english abstract.\\
% hier werden die englische Schlagw�rter aus Metadaten �bernommen
%\dckeywordsen				
%\end{abstract}

%-deutsche Zusammenfassung----------------------------------------

%\selectlanguage{german}

\begin{abstract}
\setcounter{page}{2} % Nach Bedarf anpassen!
Neurons display lots of different types of responses to synaptic inputs. Of special interest are those neurons that exhibit a bistability between spiking and resting behavior. When these are subjected to noise, the neurons perform constant transitions between both types of activity, leading to a bistability in the firing rate. Concluding from comparable systems, one can expect a region between two critical points where the reduction of noise results in a giant enhancement of spike count diffusion(Giant Diffusion). If the neuron is under the influence of a periodic signal, a similar improvement of signal transmission may occur near the critical point.\\
During the work for this thesis, three different bistable two-dimensional neuron models were investigated. By adding noise, transitions between the states were induced. All of the models exhibited a region of Giant Diffusion, thereby confirming the hypothesis. Furthermore, the effect of a slow periodic signal on the firing behavior was studied. For all models, the signal-to-noise ratio SNR could be improved by multiple orders of magnitude near the critical point.\\
Finally, the whole dynamics of the system could be captured by simply considering the transitions between the states. This Two-State-Theory not only confirmed the measurements with good agreement but also allowed for predictions at even lower noise intensities.\\
In the future, networks of bistable neurons can be simulated in an attempt to obtain a more realistic model. Additionally, bistable neurons can be experimentally investigated with regard to Giant Diffusion. These efforts may lead to a deeper insight into signal transmission and processing in the human brain.
% hier werden die deutsche Schlagw�rter aus Metadaten �bernommen
%\dckeywordsde
\end{abstract}
\thispagestyle{empty}
