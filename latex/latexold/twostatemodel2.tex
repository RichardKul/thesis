%\documentclass[10pt,a4paper]{article}
\documentclass[12pt,a4paper]{article}
\usepackage{graphicx,amsmath}
%\usepackage{subfigure}
\usepackage{float}
\usepackage[german]{babel}
\usepackage[utf8]{inputenc}
\setcounter{secnumdepth}{4}
\usepackage[top=2cm, bottom=2.5cm, left=3cm, right=3cm]{geometry}
\begin{document}


%\title{Bachelorarbeit}
%\author{Richard Kullmann}
%\date{02.06.2017}

\thispagestyle{empty}
%\setcounter{page}{2}
\newpage
\tableofcontents
\thispagestyle{empty}
\newpage
\pagenumbering{arabic}
\section{Zwei-Zustands-Theorie}
Im Falle geringer Rauschintensitäten nehmen die Übergänge zwischen Burst- und Ruhezustand deutlich weniger Zeit ein als die Verweildauer in dem jeweiligen Zustand. Deshalb bietet es sich an, das Modell in diesem Regime mit einer Zwei-Zustands-Theorie zu beschreiben. \\
Für die Übergangsraten wird angenommen, dass sie einer Arrhenius-Gleichung genügen:
\begin{align*}
r_{\pm}=r_{0,\pm}\text{e}^{-\frac{\Delta U_{\pm}}{D}}
\end{align*}
wobei $r_-$ die Übergangsrate vom ruhendem zum burstenden Zustand bezeichnet, und $r_+$ die Rate für den anderen Übergang. $\Delta U_{\pm}$ ist dabei die jeweilige Potentialbarriere, und $D$ die Rauschintensität.  Der Diffusionskoeffizient lässt sich aus der Feuerrate $v_0$ im burstenden Zustand und den Übergangsraten berechnen:
\begin{align*}
D_{\text{eff}}=\frac{v_0^2 r_+r_-}{(r_++r_-)^3}
\end{align*}
Zunächst gilt es herauszufinden, wo sich die Kurven der Diffusionskoeffizienten schneiden, sie also unabhängig von der Rauschintensität werden. Es ist
\begin{align*}
D_{\text{eff}}&=\frac{v_0^2r_{0,+}r_{0,-}\text{e}^{-\frac{\Delta U_++\Delta U_-}{D}}}{\left(r_{0,+}\text{e}^{\frac{-\Delta U_+}{D}}+r_{0,-}\text{e}^{\frac{-\Delta U_-}{D}}\right)^3}\\&=\frac{v_0^2r_{0,+}r_{0,-}}{\left(r_{0,+}\text{e}^{-\frac{3\Delta U_+-\Delta U_+-\Delta U_-}{3D}}+r_{0,-}\text{e}^{-\frac{3\Delta U_--\Delta U_+ -\Delta U_-}{3D}}\right)^3}\\&=\frac{v_0^2r_{0,+}r_{0,-}}{\left(r_{0,+}\text{e}^{-\frac{2\Delta U_+-\Delta U_-}{3D}}+r_{0,-}\text{e}^{-\frac{2\Delta U_--\Delta U_+}{3D}}\right)^3}
\end{align*}
Im Grenzfall $D\rightarrow 0,\Delta U_+>U_-$ verschwindet der erste Term im Nenner und man erhält:
\begin{align*}
D_{\text{eff}}=\frac{v_0^2r_{0,+}}{r_{0,-}^2}\text{e}^{-\frac{\Delta U_+-2\Delta U_-}{D}}
\end{align*}
Unter der Annahme, dass sich die Vorfaktoren langsam im Vergleich zu der e-Funktion verändern, ergibt sich daraus folgende Bedingung:
\begin{align*}
\Delta U_+=2\Delta U_-
\end{align*}
Aufgrund der Symmetrie des Problems liefert der andere Fall $D\rightarrow 0,\Delta U_+<U_-$:
\begin{align*}
\Delta U_-=2\Delta U_+
\end{align*}
In beiden Fällen ist die eine Potentialbarriere genau doppelt so hoch wie die andere.\\
Als nächstes ist zu untersuchen, ob der Fano-Faktor ein ähnliches Verhalten aufweist. Dieser ergibt sich aus dem Quotienten von Diffusionskoeffizient und mittlerer Feuerrate $<v>$:
\begin{align*}
F=\frac{2D_{\text{eff}}}{<v>}
\end{align*}
Wenn man berücksichtigt, dass die Feuerrate im Ruhezustand null ist, lässt sich der Mittelwert mit folgender Formel ermitteln:
\begin{align*}
<v>=v_0\frac{r_-}{r_++r_-}
\end{align*}
Damit ist
\begin{align*}
F=\frac{2v_0r_+}{(r_++r_-)^2}=\frac{2v_0r_{0,+}\text{e}^{\frac{-\Delta U_+}{D}}}{\left(r_{0,+}\text{e}^{\frac{-\Delta U_+}{D}}+r_{0,-}\text{e}^{-\frac{\Delta U_-}{D}}\right)^2}
\end{align*}
Im Grenzfall $D\ll1$ gibt es erneut zwei Lösungen. Für $\Delta U_+ > \Delta U_-$:
\begin{align*}
F=\frac{2v_0r_{0,+}}{r_{0,-}^2}\text{e}^{-\frac{\Delta U_+-2\Delta U_-}{D}} \rightarrow \Delta U_+=2\Delta U_-
\end{align*}
sowie für $\Delta U_+ < \Delta U_-$:
\begin{align*}
F=\frac{2v_0}{r_{0,+}}\text{e}^{-\frac{\Delta U_+-2\Delta U_+}{D}} \rightarrow \Delta U_+=0
\end{align*}
Der rechte Schnittpunkt verschiebt sich also nicht, während für den zweiten Schnittpunkt die eine Potentialbarriere exakt verschwinden muss.
\end{document}