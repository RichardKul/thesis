%\documentclass[10pt,a4paper]{article}
\documentclass[12pt,a4paper]{article}
\usepackage{graphicx,amsmath}
%\usepackage{subfigure}
\usepackage{float}
\usepackage[german]{babel}
\usepackage[utf8]{inputenc}
\setcounter{secnumdepth}{4}
\usepackage[top=2cm, bottom=2.5cm, left=3cm, right=3cm]{geometry}
\begin{document}


%\title{Bachelorarbeit}
%\author{Richard Kullmann}
%\date{02.06.2017}

\thispagestyle{empty}
%\setcounter{page}{2}
\newpage
\tableofcontents
\thispagestyle{empty}
\newpage
\pagenumbering{arabic}
\section{Zwei-Zustands-Theorie}
Im Falle geringer Rauschintensitäten nehmen die Übergänge zwischen Burst- und Ruhezustand deutlich weniger Zeit ein als die Verweildauer in dem jeweiligen Zustand. Deshalb bietet es sich an, das Modell in diesem Regime mit einer Zwei-Zustands-Theorie zu beschreiben. \\
Für die Übergangsraten wird angenommen, dass sie einer Arrhenius-Gleichung genügen:
\begin{align*}
r_{\pm}=r_{0,\pm}\text{e}^{-\frac{\Delta U_{\pm}}{D}}
\end{align*}
wobei $r_-$ die Übergangsrate vom ruhendem zum burstenden Zustand bezeichnet, und $r_+$ die Rate für den anderen Übergang. $\Delta U_{\pm}$ ist dabei die jeweilige Potentialbarriere, und $D$ die Rauschintensität.  Der Diffusionskoeffizient lässt sich aus der Feuerrate $v$ im burstenden Zustand und den Übergangsraten berechnen:
\begin{align*}
D_{\text{eff}}=\frac{v^2 r_+r_-}{(r_++r_-)^3}
\end{align*}
Zunächst gilt es herauszufinden, wo sich die Kurven der Diffusionskoeffizienten schneiden, sie also unabhängig von der Rauschintensität werden. Es ist
\begin{align*}
D_{\text{eff}}(\alpha_{eq})=\frac{\text{e}^{-\frac{\Delta U_++\Delta U_-}{D}}}{\left(\text{e}^{\frac{-\Delta U_m+\alpha}{D}}+\text{e}^{\frac{-\Delta U_m-\alpha}{D}}\right)^3}=\frac{\text{e}^{-\frac{2\Delta U_m}{D}}}{\text{e}^{-\frac{3\Delta U_m}{D}}\left(\text{e}^{\frac{\alpha}{D}}+\text{e}^{-\frac{\alpha}{D}}\right)^3}=c
\end{align*}
Mit $c=1$ führt dies auf:
\begin{align*}
\text{e}^{\frac{\Delta U_m}{3D}}=\text{e}^\frac{\alpha}{D}+\text{e}^{-\frac{\alpha}{D}}
\end{align*}
Im Limes $D\rightarrow 0$ ergeben sich dann zwei Schnittpunkte:
\begin{align*}
\alpha_\pm=\pm\frac{\Delta U_m}{3}
\end{align*}
sodass die eine Potentialbarriere genau doppelt so hoch ist wie die andere($\Delta U_1=2/3\Delta U_m$ sowie $\Delta U_2=4/3\Delta U_m$ und umgekehrt).\\
Als nächstes ist zu untersuchen, ob der Fano-Faktor ein ähnliches Verhalten aufweist. Dieser ergibt sich aus dem Quotienten von Diffusionskoeffizient und mittlerer Feuerrate $<v>$:
\begin{align*}
F=\frac{D_{\text{eff}}}{<v>}
\end{align*}
Wenn man berücksichtigt, dass die Feuerrate im Ruhezustand null ist, lässt sich der Mittelwert mit folgender Formel ermitteln:
\begin{align*}
<v>=v\frac{r_-}{r_++r_-}
\end{align*}
Damit ist
\begin{align*}
F(\alpha_{eq})=\frac{r_+}{(r_++r_-)^2}=\frac{\text{e}^{\frac{-\Delta U_m-\alpha}{D}}}{\text{e}^{-\frac{2\Delta U_m}{D}}\left(\text{e}^{\frac{\alpha}{D}}+\text{e}^{-\frac{\alpha}{D}}\right)^2}=\frac{\text{e}^{\frac{\Delta U_m-\alpha}{D}}}{\left(\text{e}^{\frac{\alpha}{D}}+\text{e}^{-\frac{\alpha}{D}}\right)^2}=c
\end{align*}
Wieder erhält man mit $c=1$ eine transzendente Gleichung:
\begin{align*}
\text{e}^{\frac{\Delta U_m-\alpha}{2D}}=\left(\text{e}^{\frac{\alpha}{D}}+\text{e}^{-\frac{\alpha}{D}}\right)
\end{align*}
welche im Grenzfall $D<<1$ zu zwei Lösungen führt:
\begin{align*}
\alpha_+=\frac{\Delta U_m}{3}, \qquad \alpha_-=-\Delta U_m
\end{align*}
Der linke Schnittpunkt verschiebt sich also weiter nach links, während der rechte sich nicht verändert.
\end{document}