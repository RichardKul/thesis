%\documentclass[10pt,a4paper]{article}
\documentclass[11pt,a4paper]{moderncv}
\usepackage{graphicx,amsmath}
%\usepackage{subfigure}
\usepackage{float}
\usepackage[german]{babel}
\usepackage[utf8]{inputenc}
\setcounter{secnumdepth}{4}
\usepackage[top=2cm, bottom=2.5cm, left=3cm, right=3cm]{geometry}
\usepackage{subcaption}



%\title{Bachelorarbeit}
%\author{Richard Kullmann}
%\date{02.06.2017}
\moderncvtheme[blue]{classic}
%\thispagestyle{empty}
\name{Richard}{Kullmann}
\title{Bewerbung}
\address{Alfred-Randt-Straße 34}{12559 Berlin}
\phone[mobile]{01522\,2965007}
\email{kullmann@physik.hu-berlin.de}

\begin{document}
\makecvtitle

Sehr geehrte Frau Dr. Rost-Drese,
\\\\
Am Anfang meiner Jobsuche bekam ich den Eindruck, dass ich als Physiker lediglich in der IT-Branche gute Chancen auf eine Anstellung besitze.
Daher hat mich die Aussicht auf eine vielversprechende Stelle als Mitglied in einem Forschungsteam sehr erfreut. Da ich gerade meine Masterarbeit auf dem Gebiet der Neurophysik schreibe und dementsprechend auch mehrere Veranstaltungen des Fachbereichs besucht habe, bewerbe ich mich nun auf diese Stelle.
\\
Als Kind aus einer Medizinerfamilie fand ich die biologische Anwendung von physikalischem Wissen schon immer sehr spannend. Dies hat schließlich dazu geführt, dass ich mich in meinem Masterstudium auf komplexe Systeme spezialisiert und Vorlesungen über Kinetik, Neurales Rauschen und Stochastische Systeme besucht habe, sowie mich in meiner Masterarbeit mit Nervenmodellen beschäftige. 
Auch in meiner Bachelorarbeit habe ich mich schon mit mikroskopischen Systemen beschäftigt. Bei der Untersuchung von Stickstoff-Fehlstellen in Nanodiamanten habe ich Erfahrungen im Umgang mit einem konfokalen Laser-Scanning Mikroskop gesammelt.
Des weiteren war das Programmieren ein elementarer Bestandteil des Studiums. In insgesamt drei Modulen von "Computational Physics" habe ich erfolgreich die Grundlagen des Programmierens sowie den Umgang mit Matlab gelernt. Für meine Masterarbeit verwende ich zudem Python, um einfache Berechnungen durchzuführen und Graphen zu erstellen.
\\
Neben den fachlichen Qualifikationen habe ich auch meine Fähigkeiten zur Kommunikation in verschiedenen Sprachen verbessert. Aus dem Grund, dass der Betreuer meiner Bachelorarbeit aus Italien stammte, fand die Kommunikation ausschließlich auf Englisch statt. Zusätzlich habe ich sowohl meine Bachelor- als auch Masterarbeit auf Englisch verfasst. Als letztes habe ich langjährige Erfahrung im Bereich der Wissenschaftskommunikation. Angefangen mit der Tätigkeit als Nachhilfelehrer, die bereits in der Schulzeit begann und bis heute andauert, habe ich mehrere Tutorien an der Uni gehalten und bringe seit 2018 als Vermittler des Netzwerks Teilchenwelt interessierten Schüler*innen Teilchenphysik bei. Im Rahmen dieser Tätigkeiten habe ich zudem mehrere Workshops absolviert, um meine kommunikativen Fähigkeiten zu verbessern.
\\
Bis zum Ende des aktuellen Semesters werde ich meine Masterarbeit abgeschlossen haben. Damit stehe ich Ihnen ab dem 1. April mit voller Motivation zur Verfügung. Ich danke Ihnen für die Sichtung meiner Bewerbung und freue mich über eine Einladung zum persönlichen Gespräch.
\\\\
Anlagen:
Lebenslauf,
Zeugnisse
\\\\
Mit besten Grüßen,
\\\\\\\\\\


Richard Kullmann\\
Berlin, \today
\end{document}