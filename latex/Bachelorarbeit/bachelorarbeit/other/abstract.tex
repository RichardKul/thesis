%-englische-Zusammenfassung---------------------------------------

%\selectlanguage{english}

%\begin{abstract}
%\setcounter{page}{2} % Nach Bedarf anpassen!
%Here is the english abstract.\\
% hier werden die englische Schlagw�rter aus Metadaten �bernommen
%\dckeywordsen				
%\end{abstract}

%-deutsche Zusammenfassung----------------------------------------

%\selectlanguage{german}

\begin{abstract}
\setcounter{page}{2} % Nach Bedarf anpassen!
The emerging field of magnetometry based on NV centers opens a variety of new experimental perspectives, including the imaging of single nuclear spins on the nanoscale. However, in order to achieve exceptionally long NV electron spin coherence times and high sensitivities, the NV spin needs to be decoupled from unwanted interactions with the environment. This can be accomplished with dynamical decoupling sequences.
\\
During the work for this thesis, multiple dynamical decoupling protocols were implemented and tested on NV centers in bulk diamond and nanodiamond. 
\\
The theoretical part covers general NV properties before treating the behaviour of a free electron spin and finally applying this on the NV center. Then, the effect of different decoupling protocols are discussed. After that, the structure and concept of the setup will be explained. In the final part, the measurements will be presented. The execution of the decoupling sequences will be demonstrated and the data will be used to extract the spectral density function of the environment.
\\
It was shown that all implemented dynamical decoupling sequences could enhance the coherence time. It was demonstrated that CPMG outperforms the other sequences on the given setup, achieving an improvement of up to a factor of 200 in the bulk diamond and 50 in nanodiamond. Finally, the examination of the spectral density functions of the spin bath gave a deeper insight in its coupling strength to the NV and its internal dynamics.\\
In the future, the limitations of the sequences will be further explored and other decoupling protocols will be tested. In addition to that, a better time and phase control has to be accomplished. These efforts will eventually lead to sensitivities high enough to detect small spin ensembles and even single molecular spins.
% hier werden die deutsche Schlagw�rter aus Metadaten �bernommen
%\dckeywordsde
\end{abstract}
\thispagestyle{empty}
