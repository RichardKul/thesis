\chapter{Zusammenfassung}

Es ist der Algorithmus zur Berechnung von sechsdimensionalen 
Wechselwirkungsintegralen, präsentiert in \cite{av:1a} für 
Elektronenstrukturrechnungen, überprüft, zum Teil berichtigt 
und an ein System von zwei ultrakalten Atomen angepasst 
worden. Zwei Näherungsfunktionen für realistische 
Wechselwirkungspotentiale sind vorgestellt und die Anwendung 
des Algorithmus, das heißt die Berechnung der Basisintegrale 
unter Benutzung der Näherungsfunktionen, ist analytisch 
vorbereitet worden. \\
Der Algorithmus ist in der Programmiersprache FORTRAN als 
Subroutine \texttt{gto\_int} implementiert und getestet 
worden. Dabei kann gezeigt werden, dass \texttt{gto\_int} für 
reguläre Potentiale, das heißt Integrale, die nicht renomiert 
werden müssen, in einem gewissen Parameterbereich präzise 
Ergebnisse liefert. Bei Anwendung des Renormierungsschemas 
zeigt sich jedoch ein  eingeschränkter Anwendungsbereich, 
falls x<1 ist. 
Weiterhin sind Vorschläge zur Optimierung des 
Programms angebracht worden.\\

Der unmittelbar nächste Schritt sollte es sein, das hier 
beschriebene Programm 
im Rahmen einer CI Rechnung auf das echte physikalische 
System mithilfe des nicht zu renomierenden Morse-Potentials 
anzuwenden und zu testen ob ein mit \cite{phdthesis:sergey} 
vergleichbares Energiespektrum berechenbar ist. Falls dabei 
festgestellt werden sollte, dass der 
Anwendungsbereich aktuell nicht groß genug ist, dann sollten  
relevante Optimierungen umgesetzt und erneut getestet werden. 
Weiterhin ist noch nicht geklärt, inwiefern das 
Renormierungsschema bei Verwendung des 
Lennard-Jones-Potentials konvergiert und ob damit eventuell 
ein vergleichbares, besseres oder schlechteres Ergebnis 
erzielt werden kann. Überprüft werden sollte dies, da das LJP 
prinzipiell die physikalische Gegebenheit für größere 
Abstände der Atome besser beschreibt. 
Weiterhin lässt \texttt{gto\_int} im Rahmen einer CI Rechnung 
auch mehr als nur 2 
Teilchen / Atome zu. Daher kann, falls das Zweiteilchensystem 
zu plausiblen Ergebnissen kommt, das Programm auf größere 
Systeme angewandt werden. 

%kann gezeigt werden
%zeigt sich