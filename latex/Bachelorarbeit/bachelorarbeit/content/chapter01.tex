\chapter{Einleitung}

Nach der ersten experimentellen Realisierung von Bose-Einstein-Kondensaten 
(\textit{engl.: Bose-Einstein-Condensate}, kurz BEC) im Jahre 1995 
(\cite{av:m2},\cite{av:m3},\cite{av:m4}), ca. 70 Jahre nach der theoretischen 
Vorhersage und dem darauf 
folgenden Nobelpreis \cite{av:m1} im Jahre 2001 hat sich ein neues 
Forschungsgebiet der Atom- und Molekülphysik sowie in der Quantenoptik 
eröffnet. 

BECs sind makroskopische Quantenobjekte. Sie bestehen aus 
ultrakalten, 
neutralen, bosonischen Atomen, die quantenmechanisch überwiegend den 
Grundzustand des Systems besiedeln. Dabei ist die Wellenfunktion eines 
einzelnen Teilchens stark delokalisiert.  

Eine sehr erfolgreiche Methode BECs theoretisch zu beschreiben, liefert die 
"Meanfield-Theorie"\ bei der angenommen wird, dass sich der quantenmechanische 
Zustand des Mehrteilchensystems durch eine einzige effektive Wellenfunktion 
beschreiben lässt. Durch iterative Verfahren kann so nicht 
nur die Energie, 
sondern auch die Form des BECs abgeschätzt werden. Da neutrale Atome im 
ultrakalten Regime, das heißt im Mikro- bis Nanokelvin 
Bereich, 
nur 
auf einer sehr 
kurzen Distanz wechselwirken (s-Wellen-Streuung), stellt das sogenannte 
$\delta$-Pseudopotential eine gute Näherung dar. Es entspricht einer 
Kontaktwechselwirkung mit effektiver Streulänge.

Will man darüber hinaus eine genauere Berechnung durchführen, bietet sich 
eine Konfigurationswechselwirkungsrechnung (\textit{engl.: Configuration 
Interaction}, kurz CI) an. Dabei stellt man schnell 
fest, dass bei klassischen BECs mit einer Teilchenzahl in der Größenordnung 
von $10^{5}$ (vergleiche mit \cite{av:m2}) eine zu große Anzahl an 
Konfigurationen im System vorhanden sind und so 
die Berechnung fast unmöglich scheint.

Seit 2011 ist es möglich, einzelne Atome in 
optischen Dipol-Fallen, das heißt, mit Lasern zu 
halten \cite{av:7a} und diese zu untersuchen. Weiterhin gibt es die Variante, 
sogenannte optische Gitter zu erstellen, womit BECs sogar 3d-periodisch 
angeordnet werden können (siehe u.a. in \cite{phdthesis:Schneider}, 
\cite{phdthesis:sala}). Daher kann für solche sehr kleinen Systeme dieser 
Ansatz wieder verfolgt werden.

Das Problem ist nun jedoch, dass eine genauere Berechnung von BECs mithilfe 
des Deltapotentials im Rahmen einer CI-Rechnung in 3d nicht möglich ist. 
\cite{av:1b} zeigt, dass die Energie hierbei nicht konvergiert. Für immer 
größere Konfigurationen entsteht eine immer höhere Energie des Systems.

Es gibt zwei Herangehensweisen, um mit einer solchen Divergenz umzugehen. Zum 
einen kann versucht werden, die Berechnung zu renormieren, das heißt die 
Unendlichkeit durch eine  Re-Definition von Operatoren zu umgehen. Ein solches 
Vorgehen ist unter anderem in \cite{phdthesis:sala} erfolglos versucht worden. 
%
Ein anderer Weg wäre es, ein realistischeres Wechelwirkungspotential zu 
benutzen. Dieses setzt sich prinzipiell aus zwei Teilen 
zusammen: Einen 
abstoßenden Part, der verhindert, dass sich zwei Teilchen am exakt selben Ort 
befinde und einen anziehenden Part, der durch die 
Van-der-Waals 
Wechselwirkung beschrieben wird. Wechselwirkungspotentiale sind sehr 
kurzreichweitig.  Auf der anderen Seite werden BECs in optischen Fallen 
gehalten, die in erster Näherung durch ein harmonisches Potential beschrieben 
werden können. Das Fallenpotential ist auf Grund der makroskopischen Größe der 
Apparatur langreichweitig gegenüber der Ausdehnung des BECs.  Beide Potentiale 
sollen bei einer genaueren Berechnung berücksichtigt werden, 
was auf Grund der 
unterschiedlichen Skalen der Ausdehnung die Rechnung erschwert.

Es entstehen schwer zu berechnende Wechselwirkungsintegrale, 
die prinzipiell von 6 Koordinaten abhängen und im Allgemeinen nicht separieren.

Für das Zweiteilchenproblem in einem optischen Gitter hat man in 
\cite{phdthesis:sergey} eine analytische Lösung in Relativ- und 
Schwerpunktskoordinaten gefunden. Dieses Vorgehen kann jedoch für mehr als zwei 
Teilchen nicht mehr angewendet werden.

Im Kontext der Elektronenstrukturrechnungen von Molekülen stößt man auch auf 
solche sechsdimensionalen Wechselwirkungsintegrale. 2015 wird 
in der Arbeit 
\cite{av:1a} ein Vorgehen präsentiert, wie eine sehr allgemeine Klasse der 
Integrale erst vereinfacht und anschließend über eine Schar 
an Basisintegralen 
berechnet werden kann. \\

Ziel dieser Arbeit soll es also sein, den vorgestellten 
Algorithmus zu 
überprüfen, zu implementieren und die Anwendbarkeit auf die Berechnung von 
ultrakalten Gasen mittels CI abzuschätzen. Damit wäre ein neuer Zugang zu einer 
genaueren Berechnung und damit ein tieferes Verständnis von BECs möglich. \\


Zukünftig können BECs  in vielen Anwendungsbereichen zum Einsatz kommen. 
Vorstellbar wären Bereiche wie Quantencomputer, Quantensimulatoren oder auch 
Quantensensoren. Einige Ideen zur Nutzung von BECs im Rahmen von 
Quantencomputern können in \cite{phdthesis:Schneider} 
gefunden werden. Weiterhin 
zeigt \cite{av:m5} ein Beispiel der Anwendung für Quantensensoren.
Letztere versprechen äußerst präzise Messungen auf der Skala einzelner Atome.
Um zum Beispiel Quantensensoren zu realisieren bzw. zu 
verbessern, muss es auch 
theoretische Grundlagen und präzise Vorhersagen bzgl. des Verhaltens von BECs 
auf einer sehr kleinen Skalierung und mit hoher Genauigkeit geben.\\
\ \\[1cm]
Demzufolge erklärt diese Arbeit zunächst das betrachtete System und die 
theoretische Grundlage erklärt, worauf eine CI Rechnung 
basiert. Anschließend 
wird der Algorithmus zur Berechnung von sechsedimensionalen Integralen 
vorgestellt und an das hier betrachtete Problem angepasst. In Kapitel 3 wird 
die Implementierung erläutert und getestet. Danach wird ein 
Fazit gegeben 
und ein Ausblick auf die nächsten Schritte dargestellt.
