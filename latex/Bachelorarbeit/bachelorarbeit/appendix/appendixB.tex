\chapter{Ergänzung zu den T-Integralen}
\label{sec:AnhangB:T_Integral}
In diesem Abschnitt soll zum Einen die korrigierte Rechnung 
zur Formel \ref{eq:korr_T_1} detaillierter gezeigt werden und 
zum Anderen entscheidende Formeln für die Berechnung von 
$\omega_0$ und $\omega_1$ gegeben werden, da auch diese in 
\cite{av:1a2} einen Tippfehler enthalten. Wie oben schon 
angefangen, soll $T_1^\pm(x)$ berechnet werden. Es gilt 
\begin{align*}
T_1^\pm (x)=& \mathcal{R}\int_{0}^{\infty} \frac{dz}{z} \ 
e^{\pm z - xz^2}\\
\overset{\text{part. Integration}}{=}& 
\mathcal{R}\log(z)e^{\pm 
z-xz^2}|_0^\infty - \int_{0}^{\infty}\log(z)e^{\pm z-xz^2} 
\cdot (\pm 1-2xz) \quad,
\end{align*}
wobei nun die Randterme durch das Renormierungsschema 
entfallen. Es folgt 
\begin{align*}
T_1^\pm(x) &= \mp \int_{0}^{\infty}\log(z)e^{\pm z-xz^2}\ dz 
+ 2\cdot x\int_{0}^{\infty} z\ \log(z)e^{\pm z-xz^2} \ dz 
\quad.
\end{align*}
Wird mit 
\begin{align*}
z&=\frac{u}{\sqrt{x}}\\
dz&=\frac{1}{\sqrt{x}} \ du \\
x\cdot z^2&=u^2
\end{align*} 
substituiert, folgt
\begin{align*}
T_1^\pm(x)&=\mp \frac{1}{\sqrt{x}}\int_{0}^{\infty}du\ 
\log(\frac{u}{\sqrt{x}}) e^{\pm\frac{u}{\sqrt{x}}-u^2 }+ 
2x\frac{1}{\sqrt{x}}\int_{0}^{\infty}du\ 
\log(\frac{u}{\sqrt{x}})\cdot \frac{u}{\sqrt{x}}\cdot 
e^{\pm\frac{u}{\sqrt{x}}-u^2 }\\
&=\mp \frac{1}{\sqrt{x}}\int_{0}^{\infty}du\ 
\log(\frac{u}{\sqrt{x}}) e^{\pm\frac{u}{\sqrt{x}}-u^2 }+ 
2\int_{0}^{\infty}du\ \log(\frac{u}{\sqrt{x}})\cdot u\cdot 
e^{\pm\frac{u}{\sqrt{x}}-u^2 }\\
&=\mp \frac{1}{\sqrt{x}}\int_{0}^{\infty}du\ \log(u) 
e^{\pm\frac{u}{\sqrt{x}}-u^2 }
\pm  \frac{\log(\sqrt{x})}{\sqrt{x}}\int_{0}^{\infty}du\ 
e^{\pm\frac{u}{\sqrt{x}}-u^2 }
+ 2\int_{0}^{\infty}du\ \log(u)\cdot u\cdot 
e^{\pm\frac{u}{\sqrt{x}}-u^2 }\\
&\ \ \ \ \ -2 \log(\sqrt{x})\int_{0}^{\infty}du\cdot u\cdot 
e^{\pm\frac{u}{\sqrt{x}}-u^2 }\qquad.
\end{align*}
Im letzten Schritt wurde 
$\log(\frac{a}{b})=\log(a)-\log(b)$ für a,b$\in\mathbb{R}^+$ 
verwendet. Im Vergleich zu den Definitionen \ref{eq:def:omega} von 
$\omega_k(\pm\frac{1}{\sqrt{x}})$ ist zu sehen, dass
\begin{align*}
T_1^\pm(x)&= \mp \frac{1}{\sqrt{x}} \omega_0(\pm 
\frac{1}{\sqrt{x}}) + 2\omega_1(\pm\frac{1}{\sqrt{x}}) \\
& \pm \log(\sqrt{x})\int_{0}^{\infty}\frac{du}{\sqrt{x}}\ 
e^{\pm\frac{u}{\sqrt{x}}-u^2 }
-2 \log(\sqrt{x})\int_{0}^{\infty}du\cdot u\cdot 
e^{\pm\frac{u}{\sqrt{x}}-u^2 }
\end{align*}
gilt. Für die letzten beiden Integrale kann man abermals 
substituieren mit  
$z=\frac{u}{\sqrt{x}},\ du=\sqrt{x}dz$ und erhält
\begin{align}
\int_{0}^{\infty}dz \ e^{\pm z -x z^2} &=T_0^\pm(x)= 
\sqrt{\frac{\pi}{4x}}e^{\frac{1}{4x}}\left[1\pm 
\text{Erf}\rl{\frac{1}{2\sqrt{x}}}\right]\\
\int_{0}^{\infty}dz\ z\cdot e^{\pm z -x z^2}&=\frac{1}{2x}\pm 
\frac{\sqrt{\pi}e^{\frac{1}{4x}}}{4x^{\frac{3}{2}}}\cdot 
\left[1\pm \text{Erf}\rl{\frac{1}{2\sqrt{x}}}\right]\quad.
\end{align}
Für $T_1^\pm(x)$ folgt dann
\begin{align*}
T_1^\pm(x)&= \mp \frac{1}{\sqrt{x}} \omega_0(\pm 
\frac{1}{\sqrt{x}}) + 2\ \omega_1(\pm\frac{1}{\sqrt{x}}) \\
& \pm \log(\sqrt{x})\cdot 
\sqrt{\frac{\pi}{4x}}e^{\frac{1}{4x}}\left[1\pm 
\text{Erf}\rl{\frac{1}{2\sqrt{x}}}\right]\\
&-2\log(\sqrt{x})\cdot x\cdot \biggl\{  
\frac{1}{2x}\pm 
\frac{\sqrt{\pi}e^{\frac{1}{4x}}}{4x^{\frac{3}{2}}}\cdot 
\left[1\pm \text{Erf}\rl{\frac{1}{2\sqrt{x}}}\right]
\biggr\}\quad,
\end{align*}
was nichts anderes ist als
\begin{align}\nonumber
T_1^\pm(x)&= \mp \frac{1}{\sqrt{x}} \omega_0(\pm 
\frac{1}{\sqrt{x}}) + 2\ \omega_1(\pm\frac{1}{\sqrt{x}}) 
\\\nonumber
& \pm \log(\sqrt{x})\cdot 
\sqrt{\frac{\pi}{4x}}e^{\frac{1}{4x}}\\\nonumber
& +\log(\sqrt{x})\cdot 
\sqrt{\frac{\pi}{4x}}e^{\frac{1}{4x}}\text{Erf}\rl{\frac{1}{2\sqrt{x}}}\\\nonumber
& -\log(\sqrt{x})\\\nonumber
& \mp\log(\sqrt{x})\cdot 
\sqrt{\frac{\pi}{4x}}e^{\frac{1}{4x}}\\\nonumber
& -\log(\sqrt{x})\cdot 
\sqrt{\frac{\pi}{4x}}e^{\frac{1}{4x}}\text{Erf}\rl{\frac{1}{2\sqrt{x}}}\\
%
T_1^\pm(x)&= \mp \frac{1}{\sqrt{x}} \omega_0(\pm 
\frac{1}{\sqrt{x}}) + 2\ 
\omega_1(\pm\frac{1}{\sqrt{x}})-\frac12 \log(x)\quad.
\end{align}
$\omega_0$ und $\omega_1$ können mithilfe von
%
\begin{align}\label{omega}
\omega_k(x)=\begin{cases}
\sum_{n=0}^{\infty}\frac{(-x)^n}{n!}\frac{1}{4}\Gamma\rl{\frac{n+1}{2}}
\psi_d\rl{\frac{n+1}{2}}&,
k=0\\
\sum_{n=0}^{\infty}\frac{(-x)^n}{n!}\frac{1}{4}\Gamma\rl{\frac{n}{2}+1}
\psi_d\rl{\frac{n}{2}+1}&,
k=1\\
\end{cases}
\end{align}
%
berechnet werden. $\psi_d$ ist dabei die Diagamma-Funktion 
(vergleiche \cite{av:1a2}). Hierbei fehlt in \cite{av:1a2} im 
Vergleich zu \ref{omega} der Faktor $\frac{1}{4}$.