\chapter{Schlegel-Koeffizienten und Kugelflächenfunktionen}
\label{sec:AnhangA:Schlegel}
In diesem Abschnitt sollen relevante Definitionen und 
Konventionen zur Transformation von kartesischen Ableitungen 
in sphärische Ableitungen nach Abschnitt 
\ref{sec:Hobson_schlegel} unter Zuhilfenahme der 
Schlegel-Koeffizienten und  des Hobson-Theorems angebracht 
werden. Es sei darauf hingewiesen, dass die Bedeutung der 
hier verwendeten Variablen von den in der restlichen Arbeit abweichen.  
Zunächst ist aus \cite{av:4a} bekannt, dass die 
Transformation von sphärischen in kartesische GTOs durch 
%
\begin{align}%\nonumber
%r^{l-n}\tilde{N}(n)\cdot r^n\cdot Y^l_m\cdot e^{-ar^2} 
%&=\sum_{l_x+l_y+l_z=l}^{} c(l,m,l_x,l_y,l_z)\cdot 
%N(l_x,l_y,l_z,a)\cdot x^{l_x}y^{l_y}z^{l_z}
%\cdot e^{-ar^2}\\
%\Leftrightarrow
\tilde{N}(l,a)\cdot Y^l_m 
&=\sum_{l_x+l_y+l_z=l}^{} c(l,m,l_x,l_y,l_z)\cdot 
N(l_x,l_y,l_z,a)\cdot x^{l_x}y^{l_y}z^{l_z}
\end{align}
%
gegeben ist. Hierbei sind $\tilde{N}$ und $N$ 
Normierungskonstanten und c die Schlegel-Koeffizienten, die 
durch
%
\begin{align}\nonumber
c(l,m,l_x,l_y,l_z)&=(-1)^{\frac{|m|+m}{2}}\cdot 
\sqrt{\frac{2l+1}{4\pi}\frac{(l-|m|)!}{(l+|m|)!}}\cdot 
\frac{1}{2^ll!}\cdot \\\nonumber
&\cdot 
\sum_{p=0}^{\frac{l-|m|}{2}}\binom{l}{p}\binom{p}{j}\cdot 
\frac{(-1)^p(2l-2p)!}{(l-|m|-2p)!}\cdot\\
&\cdot 
\sum_{k=0}^{j}\binom{j}{k}\binom{|m|}{l_x-2k}(\text{sign}(m)\cdot
 i)^{|m|-l_x+2k}
\end{align}
%
%clm[l_, m_, lx_, ly_, lz_] := Module[{js, val},
%If[lx + ly + lz != l, Return[0]];
%js = (lx + ly - Abs[m])/2;
%If[IntegerQ[js] == False, Return[0]];
%val = (-1)^((Abs[m] + m)/
%2) Sqrt[(2 l + 1)/(4 Pi) (l - Abs[m])!/(l + Abs[m])!]/(2^
%l l!) 
%Sum[
%Binomial[l, i] Binomial[i, js] (-1)^
%i (2 l - 2 i)!/(l - Abs[m] - 2 i)! 
%Sum[
%Binomial[js, k] Binomial[Abs[m], 
%lx - 2 k] (I Sgn2[m])^(Abs[m] - lx + 2 k), {k, 0, js}]
%, {i,0, (l - Abs[m])/2}];
%Return[val];
%]
%
gegeben sind. Hierbei ist $i$ die komplexe Einheit und 
$j:=(l_x+l_y-|m|)/2$. Weiterhin muss gelten, dass 
\begin{enumerate}
	\item ein 
	Binomialkoeffizient mit $\binom{a}{b}=0$ ausgewertet wird, falls b>a oder 
	b<0	ist,
	\item der Koeffizient c=0 gesetzt wird, falls $j\notin\mathbb{N}_0$ 
	oder $l\neq l_x+l_y+l_z$ ist,
	\item sign definiert ist als sign(m):=$\begin{cases}
	1 & , m\geq0\\
	-1 & , m<0
	\end{cases}\qquad$ .
\end{enumerate}
%
Die komplexen Kugelflächenfunktionen $Y^l_m$ sind definiert 
durch
%
\begin{align}
Y^l_m(\phi,\theta):=\sqrt{\frac{2l+1}{4\pi}\frac{(l-|m|)!}{(l+|m|)!}}\cdot
 e^{i m \phi} \cdot P^{m}_l(\cos(\theta))
\end{align}
%
mit den assoziierten Legendre-Polynomen
%
\begin{align}
P^m_l(\cos(\theta)):=\frac{(-1)^m}{2^ll!}(1-\cos^2(\theta))^{\frac{m}{2}}\cdot
\frac{\text{d}^{l+m}}{\text{d}\cos(\theta)^{l+m}}(\cos^2(\theta)-1)^l\qquad.
\end{align}
%
Es sei auf die Eigenschaft%en 
%
\begin{align}
%P^{-|m|}_l&=(-1)^m\frac{(l-|m|)!}{(l+|m|)!} P^{|m|}_l\quad,\\
Y^{-|m|}_l&=(-1)^m \rl{Y^{|m|}_l}^*
\end{align}
%
hingewiesen wodurch nur assoziierte Legendre-Polynome mit $m\geq0$ berechnet 
werden müssen \cite{b:4a}. Wie in 
\cite{av:4a} angedeutet und mit den Autoren von \cite{av:1a} 
besprochen ist, kann die reelle Version durch die Definition 
von
%
\begin{align}
c_r(l,m,l_x,l_y,l_z):=\begin{cases}
\frac{1}{\sqrt{2}}\cdot 
\left[c(l,-|m|,l_x,l_y,l_z)+(-1)^m\cdot 
c(l,|m|,l_x,l_y,l_z)\right]& , m>0 \\
c(l,0,l_x,l_y,l_z) & , m=0 \\
\frac{i}{\sqrt{2}}\cdot 
\left[c(l,-|m|,l_x,l_y,l_z)-(-1)^m\cdot 
c(l,|m|,l_x,l_y,l_z)\right]& , m<0
\end{cases}
\end{align}
%
und unter Verwendung der soliden reellen 
Kugelflächenfunktionen
%
\begin{align}
Z_{l,m}(\text{r},\theta,\phi):&=\begin{cases}
r^l\cdot\frac{1}{\sqrt{2}}\left[Y^{-|m|}_l+(-1)^m\cdot 
Y^{|m|}_l\right] & , m>0\\
r^l \cdot Y^0_l & , m=0 \\
r^l\cdot\frac{i}{\sqrt{2}}\left[Y^{-|m|}_l-(-1)^m\cdot 
Y^{|m|}_l\right]& , m<0
\end{cases}\\
&=r^l\cdot\begin{cases}
(-1)^m \cdot  
\sqrt{2}\sqrt{\frac{2l+1}{4\pi}\frac{(l-|m|)!}{(l+|m|)!}}\cdot
P^{|m|}_l(\cos(\theta))\cdot \cos(|m|\cdot \phi)& , m>0\\
\sqrt{\frac{2l+1}{4\pi}}\cdot P^{0}_l(\cos(\theta))&, m=0\\
(-1)^m\cdot \sqrt{2} 
\sqrt{\frac{2l+1}{4\pi}\frac{(l-|m|)!}{(l+|m|)!}}\cdot P^{|m|}_l(\cos(\theta)) 
\cdot \sin(|m|\cdot \phi) &, m<0
\end{cases}
\end{align}
%
zu 
%
\begin{align}
Z_{l,m}=\sum_{l_x=0}^{l}\sum_{l_y=0}^{(l-l_x)}\sum_{l_z=0}^{(l-l_x-l_y)}c_r\cdot
 x^{l_x}\cdot y^{l_y}\cdot z^{l_z}
\end{align}
%
bestimmt werden. Die reelle Rücktransformation kann durch 
Berechnung der Überlappmatrix bewerkstelligt werden. Es 
ergibt sich Formel \ref{eq:def:c_koef}
%
\begin{align*}\nonumber
c^{-1}_r(l,m,l_x,l_y,l_z)&=\frac{2}{\Gamma\rl{\frac{l_x+l_y+l_z+l+3}{2}}} 
\ \cdot \\ 
&\cdot  \sum_{a+b+c\leq l} \ _rc_{l_xl_yl_z}^{lm}\cdot 
\Gamma\rl{\frac{1+l_x+a}{2}}\cdot 
\Gamma\rl{\frac{1+l_y+b}{2}}\cdot \Gamma\rl{\frac{1+l_z+c}{2}}
\end{align*}
%
und \ref{eq:trafo_c-1}
%
\begin{align*}
x^{l_x}y^{l_y}z^{l_z} = \sum_{l=0}^{l_{max}}\sum_{m=0}^l 
c^{-1}(l,m,l_x,l_y,l_z)\cdot r^{l_{max}-l} Z_{lm}(\textbf{r})\quad,
\end{align*}
 wobei die Summe 
$\sum_{a+b+c\leq 
l}^{}$ die Bedeutung 
$\sum_{a=0}^{l}\sum_{b=0}^{l-a}\sum_{c=0}^{l-a-b}$ hat. Weiterhin wird in 
\cite{av:9a} die Beobachtung hinzugefügt, dass $c^{-1}\equiv0$ ist, falls 
$l_\text{max}-l$ ungerade ist.  Aus 
Vollständigkeitsgründen sollte erwähnt sein, dass zur 
numerischen Berechnung von $c_r$ die Fälle für $m>0$ und 
geradem $(|m|-l_x)$, $m<0$ zu ungeradem $(|m|-l_x)$ und $m=0$ zu 
geradem $l_x$ unterschiedet werden muss. In allen anderen 
Fällen ist die Summe über k identisch Null.


  